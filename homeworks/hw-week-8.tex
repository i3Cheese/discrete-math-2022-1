% !TEX options=--shell-escape
\documentclass{article}
\usepackage{tikz,amsthm,amsmath,cancel,pgfplots,animate,multirow,unicode-math,adjustbox,booktabs,array,pst-solides3d,pst-all,pst-3dplot,color,colortbl,sets,mathtools}
\usepackage[margin=96pt]{geometry}

\usetikzlibrary{automata,positioning,calc,decorations.pathmorphing,patterns,external}
\tikzexternalize[prefix=external/]

\newtheorem{problem}{Задача}
\newtheorem{lemma}{Лемма}
\newtheorem{theorem}{Теорема}
\newtheorem{definition}{Определение}
\newtheorem{example}{Пример}
\renewcommand*{\proofname}{Доказательство}
\newcommand{\const}{\mathrm{const}}
\DeclareMathOperator{\ord}{ord}
\DeclareMathOperator{\maj}{MAJ}
\newcommand{\abs}[1]{\left\lvert#1\right\rvert}
\newcommand{\floor}[1]{\left\lfloor#1\right\rfloor}
\newcommand{\ceil}[1]{\left\lceil#1\right\rceil}
\newcommand{\fr}[1]{\left\{#1\right\}}
\newcommand{\N}{\mathbb{N}}
\newcommand{\Z}{\mathbb{Z}}
\newcommand{\Q}{\mathbb{Q}}
\newcommand{\R}{\mathbb{R}}
\newcommand{\C}{\mathbb{C}}
\newcommand{\continuum}{\mathfrak{c}}
\DeclarePairedDelimiterX\set[1]\lbrace\rbrace{\def\given{\;\delimsize\vert\;}#1}

\newcolumntype{o}{>{\columncolor{orange}}c}

\makeatletter
\newenvironment{sqcases}{%
	\matrix@check\sqcases\env@sqcases
}{%
	\endarray\right.%
}
\def\env@sqcases{%
	\let\@ifnextchar\new@ifnextchar
	\left\lbrack
	\def\arraystretch{1.2}%
	\array{@{}l@{\quad}l@{}}%
}
\makeatother

\setlength{\parindent}{0pt}
\setlength{\parskip}{5pt}
\setmainfont{CMU Serif}
\widowpenalties 1 10000
\raggedbottom

\date{5 ноября, 2022}
\title{Дискретная математика \\ \Large Домашнее задание 8}
\author{Иван Мачуговский}

\AtBeginDocument{
	\renewcommand{\setminus}{\mathbin{\backslash}}
}

\begin{document}
	\maketitle

	\begin{section}{Задача 1}
		Проверим, что выполняются три свойства отношения эквивалентности:

		\begin{enumerate}
			\item Рефлексивность:

			\begin{equation*}
				x \sim x \iff \begin{cases}
					x \le x \\
					x \le x
				\end{cases} \iff x \le x,
			\end{equation*}

			что верно в силу рефлексивности $\le$.

			\item Симметричность:

			\begin{equation*}
				x \sim y \iff \begin{cases}
					x \le y \\
					y \le x
				\end{cases} \iff \begin{cases}
					y \le x \\
					x \le y
				\end{cases} \iff y \sim x.
			\end{equation*}

			\item Транзитивность:

			\begin{equation*}
				x \sim y, y \sim z \implies \begin{cases}
					x \le y \\
					y \le x \\
					y \le z \\
					z \le y
				\end{cases}.
			\end{equation*}

			В силу транзитивности $\le$ из $x \le y, y \le z$ следует $x \le z$, а из $z \le y, y \le x$ -- $z \le x$, откуда

			\begin{equation*}
				\begin{cases}
					x \le z \\
					z \le x
				\end{cases} \implies x \sim z.
			\end{equation*}
		\end{enumerate}
	\end{section}

	\begin{section}{Задача 2}
		Всего (неупорядоченных) пар из $4$ элементов $6$. Для удобства обозначим эти четыре элемента $a, b, c, d$.

		Оказывается, все семь вариантов возможны. Приведем примеры частичных порядков на каждый из вариантов ответа. Частичный порядок будем обозначать набором неравенств, например, $a \le b \le c, b \le d$, транзитивное замыкание которых дает сам порядок.

		\begin{itemize}
			\item Ноль несравнимых пар: $a \le b \le c \le d$.
			\item Одна несравнимая пара ($c$ и $d$): $a \le b, b \le c, b \le d$.
			\item Две несравнимые пары ($a$ и $b$, $c$ и $d$): $a \le c, a \le d, b \le c, b \le d$.
			\item Три несравнимые пары ($a$ и $b$, $a$ и $c$, $a$ и $d$): $b \le c \le d$.
			\item Четыре несравнимых пары ($a$ и $c$, $a$ и $d$, $b$ и $c$, $b$ и $d$): $a \le b, c \le d$.
			\item Пять несравнимых пар (все кроме упомянутой): $a \le b$.
			\item Шесть несравнимых пар (все): $\emptyset$.
		\end{itemize}
	\end{section}

	\begin{section}{Задача 3}
		Предположим, существует изоморфизм $\varphi: \Z + \N \to \Z + \Z$. В $\Z + \N$ у $1_{\N}$ (единицы из $\N$) нет предыдущего элемента, поскольку его нет в $\N$, а если $x_{\Z} < 1_{\N}$, то и $(x+1)_{\Z} < 1_{\N}$. Но в $\Z + \Z$ у $\varphi(1_{\N})$, независимо из того, в каком из двух $\Z$ он находится, предыдущий элемент есть ($\varphi(1_{\N}) - 1$), поэтому порядки не изоморфны.
	\end{section}

	\begin{section}{Задача 4}
		Предположим, изоморфизм существует. В $\N \times \Z$ любой отрезок, правый конец которого имеет вид $(1, x)$, конечен. Но для соответствующего ему элемента $(a, b)$ из $\Z \times \Z$ такое свойство не выполняется, потому что отрезок $[(a - 1, b), (a, b)]$ бесконечен, поэтому порядки не изоморфны.
	\end{section}

	\begin{section}{Задача 5}
		\begin{lemma}
			$\Q \cap (l, r) \sim \Q \cap (L, R)$ для любых рациональных $l < r, L < R$.
		\end{lemma}

		\begin{proof}
			Легко видеть, что

			\begin{equation*}
				\varphi: \Q \cap (l, r) \to \Q \cap (L, R), \ \varphi(x) = L + \frac{x - l}{r - l} (R - L)
			\end{equation*}

			есть изоморфизм.
		\end{proof}

		\begin{lemma}
			$\Q \sim \Q \cap (l, r)$ для любых $l < r$, где $l$ -- действительное число или $-\infty$, а $r$ -- действительное число или $+\infty$.
		\end{lemma}

		\begin{proof}
			Выберем возрастающую двунаправленную последовательность рациональных чисел $\{x_n\}_{n=-\infty}^\infty$ (формально -- возрастающее отображение $\mathbb{Z} \to \Q$), такую, что

			\begin{equation*}
				\lim_{n \to -\infty} x_n = l, \ \lim_{n \to +\infty} x_n = r.
			\end{equation*}

			Тогда, используя первую лемму и то, что любые два синглтона изоморфны, получаем

			\begin{multline*}
				\Q \sim \dots + \{-n\} + (-n, -n+1) + \{-n+1\} + \dots + \{m\} + (m, m+1) + \{m+1\} + \dots \sim \\
				\sim \dots + \{-x_n\} + (-x_n, -x_{n+1}) + \{-x_{n+1}\} + \dots + \{x_m\} + (x_m, x_{m+1}) + \{x_{m+1}\} + \dots \sim \Q \cap (l, r).
			\end{multline*}

			Под бесконечной суммой частично упорядоченных множеств предлагается понимать частично упорядоченное множество, носителель которого равен объединению носителей всех слагаемых, и со сравнением, для элементов из одного слагаемого наследуемого из соответствующего чума, а для элементов из разных слагаемых -- согласно взаимному расположению этих слагаемых в сумме. Такое определение позволяет рассматривать суммы двунаправленных рядов и более чем счетных рядов, чем мы и будем радостно пользоваться.
		\end{proof}

		\begin{lemma}
			$\N \times \Q \sim \Q$.
		\end{lemma}

		\begin{proof}
			В силу предудыщей леммы

			\begin{equation*}
				\N \times \Q \sim \Q + \Q + \Q + \dots \sim (\Q \cap (-\infty, \sqrt{2})) + (\Q \cap (\sqrt{2}, 2\sqrt{2})) + (\Q \cap (2\sqrt{2}, 3\sqrt{2})) + \dots,
			\end{equation*}

			а поскольку здесь концы всех интервалов иррациональны,

			\begin{equation*}
				(\Q \cap (-\infty, \sqrt{2})) + (\Q \cap (\sqrt{2}, 2\sqrt{2})) + (\Q \cap (2\sqrt{2}, 3\sqrt{2})) + \dots \sim \Q,
			\end{equation*}

			что завершает доказательство.
		\end{proof}
	\end{section}

	\begin{section}{Задача 6}
		Да, является.

		Легко показать, что если множества $(A, \le_A), (B, \le_B)$ вполне упорядочены, то и их декартово произведение вполне упорядочено: минимальный элемент множества $C \subseteq A \times B, C \ne \emptyset$ имеет первую координату, равную минимуму первых координат по всем элементам из $C$, а вторую координату -- равную минимуму вторых координат по элементам из $C$, имеющих совпадающую первую координату.

		Отсюда по индукции можно показать, что при фиксированном $n$ множество $(\N + \{\infty\})^n$ также вполне упорядочено.

		Далее отсюда показывается, что, позволяя себе некоторую свободу формулировок,

		\begin{equation*}
			\Gamma = \sum_{n=0}^\infty (\N + \{\infty\})^n
		\end{equation*}

		также вполне упорядочено: в силу вполне упорядоченности $\N_0$ в произвольном множестве $X \subseteq \Gamma, X \ne \emptyset$ можно выбрать подмножество элементов, обладающих минимальным $n$, а уже из них выбрать минимум можно в силу вполне упорядоченности $(\N + \{\infty\})^n$.

		Каждой бесконечной невозрастающей последовательности натуральных чисел $\{a_1, a_2, \dots\}$ можно сопоставить конечную последовательность длины $a_1$ с элементами из $\N + \{\infty\}$, где $k$-й \textit{с конца} элемент обозначает количество вхождений числа $k$ в $\{a_n\}$.

		Назовем это преобразование $\varphi$. Легко видеть, что оно инъективно, а также что

		\begin{equation*}
			\{x_n\} \le_{lex} \{y_n\} \iff \varphi(\{x_n\}) \le_\Gamma \varphi(\{y_n\}),
		\end{equation*}

		поэтому множество бесконечных невозрастающих последовательностей натуральных чисел с лексикографическим порядком изоморфно подмножеству $\Gamma$, а $\Gamma$ вполне упорядочено.
	\end{section}

	\begin{section}{Задача 7}
		Предположим, что бесконечных цепей в множестве нет.

		Утверждается, что при этом условии в любом бесконечном частично упорядоченном множестве $A$ найдется элемент $a \in A$, несравнимый с бесконечным числом элементов из $A$.

		\begin{proof}
			В самом деле, выберем из $A$ счетную последовательность $\{a_1, a_2, \dots\}$. Запустим алгоритм: начнем с пустой цепи и будем по одному пытаться добавлять в нее элементы $a_1, a_2, \dots$: элемент добавляется, если при его добавлении цепь будет оставаться цепью. Поскольку по предположению все цепи конечны, начиная с какого-то момента этот алгоритм перестанет добавлять в цепь элементы; назовем построенную цепь $S$.

			Для каждого элемента $s \in S$ обозначим

			\begin{equation*}
				A_s = \set{a \in \{a_1, a_2, \dots\} \mid a \text{ и } s \text{ несравнимы}}.
			\end{equation*}

			Их объединение -- это просто множество всех элементов из $\{a_1, a_2, \dots\}$, не попавших в $S$, а в силу конечности $S$ таковых счетно, поэтому хотя бы одно из $A_s$ счетно. Отсюда и следует, что существует элемент $s$, несравнимый с бесконечным числом элементов из $A$.
		\end{proof}

		Пусть $A$ -- бесконечное частично упорядоченное множество без бесконечной цепи. Используя эту лемму, приведем алгоритм построения бесконечной антицепи.

		По лемме найдется элемент $a_1 \in A$, несравнимый с бесконечым множеством элементов из $A$; назовем это множество $A_1$. Если $A$ не содержит бесконечных цепей, то и его подмножество $A_1$ не может содержать бесконечных цепей, поэтому лемма применима и к $A_1$, поэтому найдется элемент $a_2 \in A_1$, несравнимый с бесконечым множеством элементов из $A_1$. Повторим этот аргумент счетное число раз, в итоге получая последовательность $\{a_1, a_2, \dots\}$ и $\{A_1, A_2, \dots\}$. По построению

		\begin{equation*}
			\forall i < j: a_j \in A_{j-1} \subseteq A_{j-2} \subseteq \dots \subseteq A_i \implies \forall i < j: a_i \text{ и } a_j \text{ несравнимы},
		\end{equation*}

		поэтому построенная последовательность -- бесконечная антицепь, что и требовалось.
	\end{section}

	\begin{section}{Задача 8}
		Пусть дана какая-либо раскраска ребер описанного в условии графа. Изменим ориентацию синих ребер и забудем про цвета всех ребер. Легко видеть, что такое соответствие биективно. Тогда условие того, что для любых $x < y$ не существует одновренно красного и синего путей из $x$ в $y$ переходит в требование того, что не существует цикла, в котором ребра сначала все направлены слева направо, а потом справа налево. Назовем такой цикл \textit{обыкновенным} (я честно искал термин, еще не использующийся в теории графов для другой цели).

		\begin{lemma}
			Для полных графов отсутствие обыкновенных циклов эквивалентно ацикличности графа.
		\end{lemma}

		\begin{proof}
			В самом деле, если граф ацикличен, то и обыкновенных циклов в нем, в частности, нет. Покажем, что, наоборот, если граф содержит цикл, то он содержит и обыкновенный цикл.

			Пусть граф содержит некоторый простой цикл. Выберем из всех вершин этого цикла самую левую и запишем все ребра этого цикла начиная с нее: $v_1 \to v_2, v_2 \to v_3, \dots, v_k \to v_1$. Для каждого из этих $k$ ребер выпишем в строчку букву $R$, если ребро направлено слева направо, и $L$, если оно направлено справа налево. В силу того, что $v_1$ -- самая левая вершина, первая буква строки всегда $R$, а последняя -- всегда $L$. Если получившаяся строка выглядит как $R \dots RL \dots L$, то этот цикл обыкновенен, и получено требуемое.

			В противном случае легко видеть, что строка будет содержать подстроку $LR$. Это означает, что в каком-то месте этого цикла подряд идут три вершины $v_{i-1}, v_i, v_{i+1}$, где $v_{i-1} < v_i, v_i > v_{i+1}$, и есть ребра $v_{i-1} \to v_i, v_i \to v_{i+1}$, причем $2 < i < k-1$. Посмотрим теперь на направление ребра между $v_1$ и $v_i$. Если оно направлено как $v_1 \to v_i$, то $v_1 \to v_i \to v_{i+1} \to \dots \to v_k \to v_1$ будет циклом меньшей длины, чем рассматриваемый (но все еще хотя бы $3$). Аналогично, если $v_i \to v_1$, то $v_1 \to v_2 \to \dots \to v_i \to v_1$ будет циклом меньшей длины.

			Следовательно, для любого необыкновенного цикла найдется цикл строго меньшей длины. Поскольку бесконечную убывающую последовательность длин выстроить нельзя, в каком-то момент такая цепочка циклов оборвется на обыкновенном цикле.
		\end{proof}

		В силу леммы раскраска хорошая тогда и только тогда, когда соответствующий бесцветный граф ацикличен. Но ацикличный полный граф -- ни что иное, как полный порядок, где каждое ребро идет, например, от меньшего элемента к большему, а полный порядок на $n$ элементах однозначно задается перестановкой из $n$ чисел, коих $n!$.

		Ответ: $n!$.
	\end{section}
\end{document}
