\subsection{Семинар 3}
\begin{center}
\textbf{Задача 1}
\end{center}
$\displaystyle 9^{4}$

\begin{center}
\textbf{Задача 2}
\end{center}
$\displaystyle 9\cdotp 8\cdotp 7\cdotp 6$

\begin{center}
\textbf{Задача 3}
\end{center}
$\displaystyle \frac{C_{8}^{4}}{2^{8}}$

\begin{center}
\textbf{Задача 4}
\end{center}
$\displaystyle 900'000-5^{6}$ - всего чисел минус числа только из нечётных цифр

\begin{center}
\textbf{Задача 5}
\end{center}
$\displaystyle \frac{6\cdotp 6\cdotp 5!}{8!}$ - позиция этих 3 людей * количество перестановок этих людей * количество перестановок остальных / всего перестановок

\begin{center}
\textbf{Задача 6}
\end{center}
Пусть есть $\displaystyle 2$ операции: распечатать число и прибавить к нему 1, изначально число равно $\displaystyle 1$. Тогда нам нужно расставить $\displaystyle 4$ распечатывания и $\displaystyle 8$ увеличений (после каждой печати обязательно должно следовать +1, +1 может быть до 1 числа или после последнего). Итого нам нужно расставить $\displaystyle 4$ операции $\displaystyle print\ and\ +$ и 5 операций $\displaystyle +$ (так как после вывода последнего числа прибавлять не надо), это делается $\displaystyle C_{9}^{4}$ способами.

\begin{center}
\textbf{Задача 7}
\end{center}
Расставим 12 человек в ряд и сделаем пары $\displaystyle 1-2$, $\displaystyle 3-4$... Тогда каждое паросочетание посчитается $\displaystyle 6!\cdotp 2^{6}$ раз. Итого ответ $\displaystyle \frac{12!}{6!\cdotp 2^{6}}$

\begin{center}
\textbf{Задача 8}
\end{center}
Равносильно количеству решений уравнения $\displaystyle x_{1} +x_{2} +...+x_{7} =8$ (по одной монете раздали изначально). А количество таких способов: $\displaystyle C_{7+8-1}^{8}$.

\begin{center}
\textbf{Задача 9}
\end{center}
Нужно посчитать количество бинарных строк длины 15, где $\displaystyle 0$ означает что мы не взяли число, а 1 - что взяли. При этом единиц должно быть 6 и после всех единиц кроме последней обязательно должен стоять 0. То есть нам нужно расставить 4 нуля и 6 комбинаций $\displaystyle 01$ (последняя просто $\displaystyle 1$). Это можно сделать $\displaystyle C_{10}^{6} =210$ способами. А всего комбинаций $\displaystyle С_{15}^{6}$, поэтому вероятность $\displaystyle \frac{210}{С_{15}^{6}}$.

\begin{center}
\textbf{Задача 10}
\end{center}
а) Справа - количество способов выбрать капитана и добрать ему команду. Слева - устанавливаем размер команды, выбираем её и среди них выбираем капитана. Значит равнозначно.

б) todo

\begin{center}
\textbf{Задача 11}
\end{center}
В последовательности либо последний символ 0, либо последний 1, тогда обязательно перед ним 0. Получается количество последовательностей длины $\displaystyle n$ равно количеству длины $\displaystyle n-1$ (последняя 1) плюс количество длины $\displaystyle n-2$ (последний 0). База очевидна.

\begin{center}
\textbf{Задача 12}
\end{center}
todo

\begin{center}
\textbf{Задача 13}
\end{center}
Посмотрим на последний столбец. Если в нём вертикальная доминошка, то += количество способов заполнить $\displaystyle n-1$ стоблец. Если там горизонтальная, то под ним тоже обязательно горизонтальная, значит количество способов += количество способов заполнить $\displaystyle n-2$. База очевидна.

\begin{center}
\textbf{Задача 16*}
\end{center}
Одинаково, https://www.problems.ru/view\_problem\_details\_new.php?id=34899
