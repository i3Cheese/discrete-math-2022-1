\subsection{Семинар 7}
\begin{center}
\textbf{Задача 1}
\end{center}
а) $\displaystyle R\sim \{0,1\}^{\mathbb{N}} \sim \mathbb{N}^{\mathbb{N}}$.

Первая эквивалентность по определению, доказательство второй:

Последовательность$\displaystyle \leftrightarrow $$\displaystyle \mathbb{N}^{\mathbb{N}}$: количество подряд идущих единиц - число, $\displaystyle 0$ - разделитель.



б) Пронумеруем $\displaystyle \mathbb{Q}$ и заменим элементы на индексы. Получили прошлый пункт

\begin{center}
\textbf{Задача 2}
\end{center}
Докажем, что количество прямых равносильно $\displaystyle \mathbb{R}^{4}$:

Из прямой в набор: выберем 2 разные точки на прямой и запишем их координаты. У разных прямых будут разные координаты.

Обратно; $\displaystyle \mathbb{R}^{4} \sim \mathbb{R}$, поэтому для $\displaystyle y\in \mathbb{R}$ проведём $\displaystyle f=y$.

\begin{center}
\textbf{Задача 3}
\end{center}
Докажем, что $\displaystyle \mathbb{R} \sim \mathbb{R}^{2}$, дальше доказательство рекурсивно.

$\displaystyle \mathbb{R} \sim ( 0,1) \sim ( 0,1)^{2}$ (каждое число возводим в квадрат) $\displaystyle \sim \mathbb{R}^{2}$.

\begin{center}
\textbf{Задача 4}
\end{center}
Доказать что $\displaystyle \mathbb{R}^{\mathbb{N}} \sim \mathbb{R} \ \Leftrightarrow \ \{0,1\}^{\mathbb{N}^{\mathbb{N}}} \sim \{0,1\}^{\mathbb{N}}$

Это верно, так как распишем последовательности в таблицу и будем проходить по ней по диагоналям, получим одну последовательность, по которой можно восстановить изначальную.

\begin{center}
\textbf{Задача 5}
\end{center}
а) Конечно (не конечно в смысле оно точно счётно, а правда оно конечно), так как первые $\displaystyle k$ элементов задают всю последовательность (она периодична с периодом $\displaystyle k$)

б) Из двоичной последовательности в $\displaystyle \{0,1\}^{\mathbb{N}}$ - очевидно.

В обратную сторону: из $\displaystyle a_{n}$ сделаем $\displaystyle [ 0,1,a_{1} ,0,1,a_{2} ,...]$. Все условия выполняются вне зависимости от значений $\displaystyle a_{n}$.

\begin{center}
\textbf{Задача 6}
\end{center}
Выберем внутри обеих окружностей рациональные точки. У непересекающихся восьмёрок они не могут пересекаться, значит подмножество (инъекция) $\displaystyle \mathbb{Q}^{4} \sim \mathbb{N}$.

\begin{center}
\textbf{Задача 7}
\end{center}
Да, горизонтальные прямые на плоскости. Их континуум (по $\displaystyle y$ координате), и их объединение равно $\displaystyle \mathbb{R}^{2} \sim \mathbb{R}$.

\begin{center}
\textbf{Задача 8}
\end{center}
todo