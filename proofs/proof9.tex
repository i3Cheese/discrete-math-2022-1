\subsection{Полнота системы связок «XOR, конъюнкция, 1». Теорема о представлении булевой функции полиномом Жегалкина (существование и единственность).}
\textbf{Полнота системы связок многочлена Жегалкина.}: Система связок $\{1, \oplus, \wedge\}$ -- полная система связок\\

\noindent \textbf{Доказательство:}\\

Индукция по $n$:\\

База: $n = 0$ 

0 = $1 \oplus 1$, $1 = 1$\\

Переход: $n \rightarrow n + 1$\\

$f(x_1,x_2,\ldots,x_n,x_{n+1}) = (f(x_1,\ldots,x_n,0) \wedge (x_{n+1} \oplus 1)) \oplus (f(x_1, \ldots, x_n, 1) \wedge x_{n+1})$\\

Подставляем вместо $x_{n+1}$ 0 и 1, получаем функци уже от $n$ переменных. По индукции, для них уже построены многочлены Жегалкина. Потому, подставим их вместо функций и приведем подобные, получим многочлен Жегалкина от $n + 1$ переменной.\\

\textbf{Теорема о представлении булевой функции полиномом Жегалкина (существование и единственность).}\\

Всякая булева функция от $n$ переменных единственным образом представима в виде полинома Жегалкина(с точностью до перестановки мономов и переменных).\\

\noindent \textbf{Доказательство:}\\

Существование было доказано только что.\\

Теперь докажем единственность. Всего булевых функций от $n$ переменных - $2^{2^n}$. Количество мономов($a_i$) в полиноме Жегалкина - $2^n$, каждый
может быть равен 0 или 1. Значит всего полиномов Жегалкина - $2^{2^n}$.

Теперь построим биекцию $f : \{\Pi_1, ..., \Pi_{2^{2^n}} \} \to \{f_1, ..., f_{2^{2^n}}\}$, где $f_i$ - булева функция, а $\Pi_i$ - полином Жегалкина.

Пусть $f(\Pi_i)$ - функция, которой соответствует полином Жегалкина. Тогда $f$ - сюръекция и тотальная функция, но в силу равенства количества
функций и полиномов Жегалкина, $f$ - инъекция, поэтому $f$ - биекция. Значит представление функции в виде полинома Жегалкина единственно. Чтд
