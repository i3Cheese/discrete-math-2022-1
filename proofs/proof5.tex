\subsection{Числа сочетаний: явная и рекуррентная формула. Треугольник Паскаля. Рекуррентное соотношение для чисел сочетаний. Бином Ньютона. Сумма биномиальных коэффициентов. Знакопеременная сумма биномиальных коэффициентов.}
\begin{itemize}
	\item Число размещений
	$$A^k_n = \frac{n!}{(n-k)!}$$
	Количество способов извлечь первый элемент равно $n$. Удалим его из множества. Количество затем извлечь из оставшихся $n-1$ объекта второй элемент равно $n - 1$. Продолжим эту процедуру пока не извлечем $k$ элементов. (на последнем шаге количество способов будет равно $n - k + 1$). Применим правило умножения и получим, что количество способов извлечь $k$ произвольных элементов (что и есть неупорядоченный $k$-элеметный набор элементов) из $n$-элементого множества равно $A^k_n = n \cdot (n - 1) \cdot (n - 2) \cdot ... \cdot (n - k + 1) = \frac{n!}{(n-k)!}$.
	
	\item Число сочетаний явная формула
	$$C^k_n = \frac{n!}{(n-k)!k!}$$
	Выпишем все сочетания. Далее заменим каждое на всевозможные его перестановки (то есть $\{a_1, a_2, ..., a_{k-1},a_k\} \to \{\{a_1, a_2, ..., a_{k-1},a_k\}, \{a_1, a_2, ..., a_{k}, a_{k-1}\},...,\{a_k, a_{k-1}, ..., a_1\}\}$). Получим всевозможные размещения. Количество способов переставить сочетание размера $k$ равно $k!$ (так как все элементы различные). Таким образом получим $A^k_n = C^k_n \cdot k! \Rightarrow C^k_n = \frac{A^k_n}{k!} = \frac{n!}{(n-k)!k!}$.
	
	\item Число сочетаний рекуррентная формула
	$$C^k_n = C^{k-1}_{n-1} + C^{k}_{n-1}$$
	Обозначим первый элемент $n$-элементого множества за $a$. Любое сочетание размера $k$ из этого множества либо содержит его, либо не содержит. Число сочетаний размера $k$ не содержащих $a$ равно числу сочетаний размера $k$ из $(n-1)$-элементного множества, то есть $C^k_{n-1}$. Число сочетаний размера $k$ содержащих $a$ равно числу сочетаний размера $(k - 1)$ из $(n - 1)$-элементого множества, то есть $C^{k-1}_{n-1}$. В итоге получим:
	$$C^k_n = C^{k-1}_{n-1} + C^{k-1}_{n}$$
	
	\item Бином Ньютона
	$$(a + b)^n = \sum_{k=0}^{n}C^k_na^kb^{n-k}$$
	Очевидно, что $(a + b)^n = \underbrace{(a+b)(a+b)...(a+b)}_{n}$. Пусть мы взяли $a$ из $k$ скобок и $b$ из остальных $n - k$ скобок. Получим слагаемое вида $a^kb^{n-k}$. Количество способов взять такое слагаемое равно количеству способов выбрать $k$ скобок из которых мы возьмем $a$ (так как если нам известно из каких скобок мы возьмем $a$, нам известно из каких скобок мы возьмем $b$). Это количество равно числу сочетаний размера $k$ из $n$-элементного множества, то есть слагаемое $a^kb^{n-k}$ войдет в итоговое разложение $C^k_n$ раз. Получим:
	$$(a+b)^n = C^0_n \cdot a^0 \cdot b^n + C^1_n \cdot a^1 \cdot b^{n-1} + ... + C^n_n \cdot a^n \cdot b^0 = \sum_{k=0}^{n}C^k_na^kb^{n-k}$$
	
	\item Сумма биномиальных коэффициентов
	$$\sum_{k=0}^{n}C^k_n = 1^0 \cdot 1^n \cdot C^0_n + 1^1 \cdot 1^{n-1} \cdot C^1_n + ... + 1^n \cdot 1^0 \cdot C^n_n  = (1 + 1)^n = 2^n$$
	
	\item Знакопеременная сумма биномиальных коэффициентов
	$$\sum_{k=0}^{n}(-1)^kC^k = (-1)^0 \cdot 1^n \cdot C^0_n + (-1)^1 \cdot 1^{n-1} \cdot C^1_n + ... + (-1)^n \cdot 1^0 \cdot C^n_n  = (-1 + 1)^n = 0^n = 0$$

\end{itemize}