\subsection{Связь длины цепей и размеров разбиений частично упорядоченного множества на антицепи.}

Пусть нам дано конечное частично упорядоченное множество $(P, \leq)$. Напомню, что $c_{\max}$ - наибольший размер цепи, $\hat{a}$ - наименьший размер разбиения на антицепи.
Теперь мы готовы доказывать теорему.\\

\noindent \textbf{Доказательство.}\\

Сначала докажем, что $c_{\max} \leq \hat{a}$. Разложим наше множество на минимальное количество антицепей, из каждой мы можем выбрать максимум по 1 элементу,
чтобы получить цепь, значит по принципу Дирихле наше неравенство верно.\\

Теперь докажем обратное, что $c_{\max} \geq \hat{a}$.

\begin{center}
    Возьмем $P_1 = \min (P)$ - множество всех минимальных элементов $P, \min (P) \neq \varnothing$.\\
    $P_2 = \min(P \backslash P_1)$\\
    $P_3 = \min(P \backslash (P_1 \cup P_2))$\\
    $\dots$\\
    $P_n = \min(P \backslash (P_1 \cup ... \cup P_{n - 1}))$\\
    $P_{n + 1} = \varnothing$ - такое $n + 1$ найдется, так как множество конечно.
\end{center}

Все такие слои - антицепи. Возьмем $p_n \in P_n$. Заметим, что $p_n$ - не минимальный в $P_{n - 1} \cup P_n \Rightarrow \exists p_{n - 1} \in P_{n - 1} : p_n > p_{n - 1}$.
$p_{n - 1}$ - не минимальный в $P_n \cup P_{n - 1} \cup P_{n - 2} \Rightarrow \exists p_{n - 2} \in P_{n - 2} : p_n > p_{n - 1} > p_{n - 2}$.

Продолжаем этот процесс, пока не получим цепь $c$, $p_n > p_{n - 1} > ... > p_{n - 2} > ... > p_1$.\\

Значит $c_{\max} \geq |c| = n \geq \hat{a}$. Значит $c_{\max} = \hat{a}$. Чтд
