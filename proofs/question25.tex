\subsection{Доказательство того, что достижимость в неориентированном графе является отношением эквивалентности и всякий граф можно разбить на компоненты связности. Неравенство, связывающее число вершин, ребер и компонент связности в графе. Разбиение ориентированного графа на компоненты сильной связности.}
\begin{itemize}
	\item \textbf{Доказательство того, что достижимость в неориентированном графе является отношением эквивалентности и всякий граф можно разбить на компоненты связности}

		\begin{itemize}
		    \item
		        Отношение связности считается отношением эквивалентности, если выполнены три условия:
		        \begin{enumerate}
		        \item Рефлексивность: $xRx \; \forall x \in V$
		        \item Симметричность: $xRy \Rightarrow yRx \; \forall x, y \in V$
		        \item Транзитивность: $xRy, yRz \; \Rightarrow xRz \; \forall x, y, z \in V$
		        \end{enumerate}
		    \item
		    Докажем все эти 3 свойства.
		        \begin{enumerate}
		        \item
		        Рефлексивность: между вершинами $x$ и $x$ всегда существует путь нулевой длины, поэтому они лежат в одной компоненте связности
		
		        \item
		        Симметричность: если можно попасть из вершины $x$ в вершину $y$, то можно попасть из вершины $y$ в вершину $x$ просто пройдя по всем ребрам в обратном порядке.
		
		        \item
		        Транзитивность: если есть путь из вершины $x$ в вершину $y$, а из вершины $y$ - в вершину $z$, то "склеив" \, эти два пути, можно получить путь из вершины $x$ в вершину $y$
		
		        \end{enumerate}
		    \item
		    Таким образом, наше множество вершин $V$ делится на классы эквивалентности, называемые компонентами связности
		
		\end{itemize}
	
	\item \textbf{Неравенство, связывающее число вершин, ребер и компонент связности в графе}
	
		\begin{itemize}
			\item
			Кол-во компонент связности в графе $G$ всегда больше, либо равно $(|V| - |E|)$, где $V$ - множество вершин в графе $G$, $E$ - множество ребер в графе $G$.
			
			\item
			Докажем это. Для начала, удалим все ребра из нашего графе, после чего будем их последовательно добавлять и следить за значением двух величин: разностью $(|V| - |E|)$ и количеством компонент связности. Каждый раз, когда мы добавляем ребро, $|E|$ увеличивается на один, соответственно, разница $(|V| - |E|)$ уменьшается на один. Посмотрим на концы добавленного ребра. Если эти вершины были в одной компоненте связности, то количество компонент связности не изменится. Иначе, две компоненты связности соединятся, и общее количество компонент связности уменьшится на 1. Выходит, после каждого шага $(|V| - |E|)$ уменьшается на 1, а кол-во компонент связности уменьшается либо на 0, либо на 1. Стоит отметить, что до добавления первого ребра эти две величины будут равны. Отсюда следует инвариант: после добавления любого ребра $(|V| - |E|)$ будет больше, либо равно кол-ва компонент связности.
		\end{itemize}
	
	\item \textbf{Разбиение ориентированного графа на компоненты сильной связности}
	
	\begin{itemize}
	\item
	Отношение сильной связности - отношение эквивалентности.
	
	\item
	Докажем это.
	\begin{enumerate}
	\item Рефлексивность: между вершинами $x$ и $x$ всегда существует путь нулевой длины, поэтому они лежат в одной компоненте связности
	\item Симметричность: если вершины $x$ и $y$ сильно связаны, то по определению существует ориентированный путь из $x$ в $y$ и из $y$ в $x$.
	\item Транзитивность: если вершины $x$ и $y$, $y$ и $z$ сильно связаны, то по определению есть ориентированный путь $x \to y$ и $y \to z$, следовательно, есть ориентированный путь $x \to z$. Аналогично показывается, что есть путь $z \to x$.
	\end{enumerate}
	
	\item
	Наше множество вершин в графе $G$ разбивается на непересекающиеся подмножества вершин: $V_1, V_2, \dots, V_k$. Они называются компонентами сильной связности. Сожмем каждую компоненту в одну вершину. Затем, посмотрим на наши исходные ребра в нашем графе. Если концы i-ого ребра находятся в разных компонентах связности, соединим две вершины, которые будут соответствовать сжатым компонентам связности. Получим новый граф, который называется \textbf{конденсацией графа G}.
	\end{itemize}

\end{itemize}