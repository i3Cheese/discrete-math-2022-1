\subsection{Применения метода математической индукции: существование 2-цветной раскраски областей на плоскости; неравенство Бернулли; сумма обратных квадратов меньше 2}
	\begin{itemize}
		\item Существование 2-цветной раскраски областей на плоскости
		\begin{itemize}
			\item Утверждение: $n$ прямых делят плоскость на области. $A(n)$ - верно ли, что эти области можно раскрасить в 2 цвета так, чтобы никакие две соседние области не были покрашены в один цвет.
			\item База:
			\item Шаг: пусть $A(n)$ - верно, докажем верность $A(n + 1)$:
			
			По сути нам дана правильная раскраска плоскости в случае $n$ прямых. Утверждается, что если при добавлении $n+1$ прямой инвертировать цвет всех областей по одну сторону от нее, то мы получим правильную раскраску. Докажем, что любая граница разделяет области разных цветов. Для этого рассмотрим $2$ случая:
			\begin{enumerate}
				\item Граница принадлежит какой-либо из старых $n$ прямых. Тогда области, которые она разделяет, лежат по одну сторону от новой прямой. Поэтому поскольку старая раскраска была правильной, то в новой они также будут разного цвета.
				\item Граница принадлежит новой $n+1$ прямой. Тогда области, что она разделяет, в старой раскраске были одного цвета, мы инвертируем только одну из них, поэтому получаем $2$ разных цвета.
			\end{enumerate}
			
			Таким образом $A(n + 1)$ верно $\Rightarrow$ индукция верна $\Rightarrow$ исходное утверждение верно.
		\end{itemize}
		
		\item Неравенство Бернулли
		\begin{itemize}
			\item Утверждение: $A(n)$ - верно ли, что $(1 + x)^n \geqslant 1 + xn, \: x \in \R, x > -1$
			\item База: $A(1) \colon (1 + x)^1 \geqslant 1 + x\cdot1 \Leftrightarrow 0 \geqslant 0 \Rightarrow$ база верна
			\item Шаг: пусть $A(n)$ - верно, докажем верность $A(n + 1)$:
			$$(1 + x)^{n + 1} = (1 + x)^n(1 + x) \geqslant (1 + xn)(1 + x)\geqslant$$$$ \geqslant (1 + xn) + x = 1 + x(n + 1)$$
			Таким образом $A(n + 1)$ верно $\Rightarrow$ индукция верна $\Rightarrow$ исходное утверждение верно.
		\end{itemize}
		\item Сумма обратных квадратов меньше 2
		\begin{itemize}
			\item Утверждение: $A(n)$ - верно ли, что $\displaystyle\sum_{k=1}^{n} \frac{1}{k^2} \leqslant 2 - \frac{1}{n}$
			\item База: $A(1)\colon \displaystyle\sum_{k=1}^{1}\frac{1}{k^2} = \frac{1}{1} = 1 \leqslant 2 - 1 \Rightarrow$ база верна
			\item Шаг: пусть $A(n)$ - верно, докажем верность $A(n + 1)$:
			$$\displaystyle\sum_{k=1}^{n+1} \frac{1}{k^2}  \leqslant 2 - \frac{1}{n} + \frac{1}{n+1} = 2 - \frac{n^2 + n + 1}{n(n + 1)^2} \leqslant 2 - \frac{n(n + 1)}{n(n+1)^2} = 2 - \frac{1}{n+1}$$
			Таким образом $A(n + 1)$ верно $\Rightarrow$ индукция верна $\Rightarrow$ исходное утверждение верно. Так как $\frac{1}{n} > 0$ получаем:
			$$\displaystyle\sum_{k=1}^{n} \frac{1}{k^2} \leqslant 2 - \frac{1}{n} < 2$$
		\end{itemize} 
	\end{itemize}
