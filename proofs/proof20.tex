\subsection{Изоморфизм конечных линейных порядков одинаковой мощности. Теорема о том, что счетный линейный порядок изоморфен подмножеству рациональных чисел.}
\begin{itemize}
	\item \textbf{Изоморфизм конечных линейных порядков одинаковой мощности}
	
	Пусть $(P, \leqslant_P), (Q, \leqslant_Q)$ - конечные линейно упорядоченные множества и $|P| = |Q|$. Тогда $(P, \leqslant_P) \cong (Q, \leqslant_Q)$ 
	
	Доказательство:
	
	Докажем по индукции. База - один элемент в порядке, тривиально. Предположим, что все линейные порядки мощности $n$ изоморфны. Рассмотрим произвольные линейные порядки $P$ и $Q$ с $(n+1)$ элементов. Выделим в них наименьшие элементы $p$ и $q$ (они существуют, так как линейные порядки конечны). Порядки на оставшихся $n$ элементах изоморфны по предположению индукции. Поставив в соответствие $p$ элемент $q$ получим изоморфизм порядков $P$ и $Q$.
	
	\item \textbf{Теорема о том, что счетный линейный порядок изоморфен подмножеству рациональных чисел}
	
	Пусть $(P, \leqslant_P)$ - счетное линейное упорядоченное множество. Тогда $(P, \leqslant_P) \cong (A, \leqslant)$, где $A \subseteq \Q$
	
	Доказательство:
	
	Пусть $P = \{p_1, p_2, p_3, ...\}$
	
	Построим инъекцию $\phi \colon P \to \Q$ : $p_i \leqslant_P p_j \Leftrightarrow \phi(p_i) \leqslant \phi(p_j)$
	
	$\phi(p_1) = 1$
	
	Пусть определены значения функции в точках $p_1, p_2, ..., p_n$. Определим $\phi(p_{n+1})$:
	
	\begin{enumerate}
		\item Если $p_{n+1}$ - наибольщий элемент в множестве $\{p_1, p_2, ..., p_n, p_{n+1}\}$, тогда $\phi(p_{n+1}) = \max_{1 \leqslant i \leqslant n}\phi(p_i) + 1$
		\item Если $p_{n+1}$ - наибольщий элемент в множестве $\{p_1, p_2, ..., p_n, p_{n+1}\}$, тогда $\phi(p_{n+1}) = \min_{1 \leqslant i \leqslant n}\phi(p_i) - 1$
		\item Если $p_i < p_{n+1} < p_j$, где $|[p_i, p_j]| = 3$, тогда $\phi(p_{n+1}) = \frac{\phi{p_i} + \phi{p_j}}{2}$
	\end{enumerate}
	
	Получили инъекцию, значит $P \cong \phi{P} = A$, ч.т.д.
	
\end{itemize}