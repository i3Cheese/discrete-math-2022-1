\subsection{Принцип математической индукции. Принцип полной математической индукции. Принцип наименьшего числа.}{
	\begin{itemize}
		\item \textbf{Принцип математической индукции:}
		
		Пусть есть некоторое утверждение $A$ зависящее от $n \in \N$, которое может быть либо верным, либо ложным, и выполняются следующие условия:
		\begin{enumerate}
			\item $A(1)$ верно (База индукции)
			\item $\forall n : A(n)$ - верно $\Rightarrow A(n + 1)$ верно. (Шаг индукции)
		\end{enumerate}
		То $\forall n : A(n)$ - верно.
		
		\item \textbf{Принцип математической индукции (эквивалентная формулировка):}
		
		Пусть $S \subseteq \N$ и выполняются следующие условия:
		\begin{enumerate}
			\item $1 \in S$
			\item $\forall n \in \N : n \in S \Rightarrow n + 1 \in S$
		\end{enumerate}
		Тогда $S = \N$.
		
		\item \textbf{Принцип полной математической индукции:}
		
		Пусть есть некоторое утверждение $A$ зависящее от $n \in \N$, которое может быть либо верным, либо ложным, и выполняются следующие условия:
		\begin{enumerate}
			\item $A(1)$ верно
			\item $\forall n : (\forall k < n \: A(k)$ - верно) $\Rightarrow A(n)$ верно.
		\end{enumerate}
		То $\forall n : A(n)$ - верно.
		
		\item \textbf{Принцип наименьшего числа}
		
		Пусть $S \subseteq \N, \: S \neq \varnothing \Rightarrow$ в $S$ существует наименьший элемент.
		
		Наименьшим элементом множества $A$ называют такое число $c$, что $\forall a \in A : c \leqslant a$
		
	\end{itemize}
}