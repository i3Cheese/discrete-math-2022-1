\subsection{Функции (как частный случай отношений). Образы и прообразы множеств. Обратная функция.}

\textbf{Функция f из A в B} -- это такое отношение $f \subseteq A \times B$, что $\forall a \in A$ в $f$ есть не более одной пары $(a, b)$, где $b \in B$.

Обозначение: $(a, b) \in f$ или $afb$ $\Leftrightarrow f(a) = b$.

Мы рассматриваем частичные функции, то есть они не полностью определены на $A$. Но

f на A и B \textbf{тотальна}, если $Dom f = A$ (функция определена на всем множестве A). Тогда пишут $f: A \rightarrow B$.

Запись $f: A \rightarrow B$ с подвохом: мы подразумеваем при подобной записи что $f$ тотальна, однако это может быть не так вне нашего курса, будьте бдительны.

Если $X \subseteq A$, то $f(X) = \{b \in B \mid \exists x \in X: f(x) = b\}$ -- \textbf{образ} множества $A$.

\textbf{Прообраз} множества $Y$ $f^{-1}(Y) (Y \subseteq B) = \{a \in A \mid f(a) \in Y\}$.

Пусть $f: A \rightarrow B$ -- биекция. Тогда $f^{-1}: B \rightarrow A$ или \textbf{обратная функция к $f$} определяется как $f^{-1}(b) = a \Leftrightarrow f(a) = b$.

Эквивалентное определение: функция $g : B \to A$ называется обратной к $f : A \to B$, если $g \circ f = id_A, f \circ g = id_B$. 
