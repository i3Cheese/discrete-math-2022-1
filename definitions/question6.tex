\subsection{Отношения эквивалентности. Классы эквивалентности.}

   Отношение $R$ на $A$ называют:

   \textbf{Рефлексивным}, если $\forall a \in A$, $aRa$.

   \textbf{Симметричным}, если $\forall a, b \in A$, ($aRb \Leftrightarrow bRa$).

   \textbf{Транзитивным}, если $\forall a, b, c$, ($aRb$ и $bRc$ $\Rightarrow$ $aRc$).

   Пример: отношение $a < b$ транзитивно, но не рефлексивно и не симметрично. Отношение $a + b = a * b$ симметрично, но не рефлексивно и не транзитивно.

   Отношение $R$ на $A$ называют \textbf{отношением эквивалентности}, если отношение $R$ рефлексивно, симметрично и транзитивно.

    Пример: Отношение $a = b$: рефлексивно ($a = a \forall a \in A$), симметрично ($a = b \Rightarrow b = a \forall a, b \in A$), транзитивно ($a = b. b = c \Rightarrow a = c \forall a, b, c \in A$).

     Если $R$ на $A$ -- отношение эквивалентности, то множество $A$ можно разбить на классы эквивалентности $A_i$

     \textbf{Классы эквивалентности} -- это разбиение множества $A$ отношением эквивалентности $R$ на непересекающиеся классы $(A_i \cap A_j = \emptyset \forall i \neq j, \vee_{i \in I} A_i = A)$ такое, что $\forall x, y \in A_i xRy$ и $\forall x \in A_i, y \in A_j, i \neq j, \neg xRy$. (то есть если два элемента принадлежат одному классу эквивалентности, они находятся в отношении $R$ и наоборот).
