\subsection{Множества, теоретико-множественные операции. Парадокс Рассела.}
\begin{itemize}
	\item \textbf{Определение и некоторые обозначения}
	
Множеством называют совокупность произвольных объектов

$X = \{a, b, c\}$

$a \in X$ -- объект $a$ лежит в множестве, $d \notin X$ -- объект $d$ не лежит в множестве

Способы задания множества:
\begin{enumerate}
	\item Явно (списком элементов): $X = \{1, 2, 3\}$
	\item Условием: $Y = \{ y \in \N \mid y$ -- четно$\}$
\end{enumerate}

$\varnothing$ -- пустое множество

$2^{A}$ -- множество всех подмножеств $A$ (в том числе пустое и само $A$)

\item \textbf{Операции над множествами}

Пусть $A, B$ -- множества. Тогда:

Объединение множеств: $A \cup B = \{x \mid x \in A \vee x \in B$ $\}$

Пересечение множеств: $A \cap B = \{x \mid x \in A \wedge x \in B$ $\}$

Разность множеств: $A \setminus B = \{x \in A \mid x \notin B \}$

Дополнение множества $A$ до $B$: $\bar{A} = B \setminus A$

Симметрическая разность: $A \triangle B = \{x \mid (x \in A \wedge x \notin B) \vee (x \notin A \wedge x \in B) \}$

$A \subseteq X \Leftrightarrow \forall x (x \in A \Rightarrow x \in X)$. $A$ -- подмножество, $X$ -- надмножество.

$A = B \Leftrightarrow A \subseteq B \wedge B \subseteq A$.

\item \textbf{Парадокс Рассела}

$U = \{x \mid x \notin x\}$.

Вопрос: $U \in U$?

Если да, то по определению $U$, $U \notin U$. Если нет, то т.к $U \notin U$, $U$ является элементом себя же. Противоречие. 

\end{itemize}