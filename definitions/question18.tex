\subsection{Класс монотонных функций, лемма о немонотонной функции.}

   Для того, чтобы ввести класс монотонных функций нам нужно ввести понятие порядка на множестве наборов переменных. Скажем, что изначально $0 < 1$. Тогда:

   \textbf{Набор $(\alpha_1, \ldots, \alpha_n)$ $\le$ $(\beta_1, \ldots, \beta_n)$}, если $\forall i \colon \alpha_i \le \beta_i$.

   Пример: $(1, 0) \le (1, 1)$

   $(1, 0) \nleq (0, 1)$ (не сравнимы)

   $(0, 1) \nleq (1, 0)$ (не сравнимы)

   \textbf{$f \in P_2$ монотонная}, если $\forall \alpha, \beta$, $\alpha_i \le \beta_i \Rightarrow f(\alpha) \le f(\beta)$

    \textbf{Лемма о немонотонной функции:}

   Пусть $f (x_1, \ldots, x_n) \notin M$. Тогда, подставляя вместо переменных 0, 1, x, можно получить $\neg x$.
