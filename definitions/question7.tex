\subsection{Правила суммы и произведения в комбинаторике. Задачи о подсчете путей. Конечные слова в алфавите. Упорядоченный выбор $k$ элементов из $n$ (с повторениями или без повторений)}

$A$, $B$, $C$ - какие-то конечные множества.\\
Правило суммы:

Пусть $A = B \cup C$ и $B \cap C = \varnothing$, тогда $|A| = |B| + |C|$\\
Правила произведения:

$|A \times B| = |A| \cdot |B|$

Задачи о подсчетах путей:

Пусть дана сеточка и вы можете ходить только в право и в лево по ее узлам. Необходимо посчитать количество путей из левого нижнего угла в правый верхний угол данной сетки.

\includegraphics[scale=1.5]{definitions/images/first-patch.jpg}

Можно сделать это, используя, правило суммы. Для каждого из узлов вычисляя количество способов дойти из левого нижнего угла в этот. Для того чтобы насчитать количество путей для очередного узла мы можем просто сложить количество путей ведущих в узел ниже и узел левее. Это верно так как два множества этих путей не пересекаются.

\includegraphics[scale=1.5]{definitions/images/second-patch.jpg}

Вторая задача - найти количество путей из $s$ в $f$ в подобном графе.

\includegraphics[scale=1.5]{definitions/images/third-patch.png}

Множество таких путей можно представить себе как декартово произведение множеств путей между парами вершин. Тогда количество путей из $s$ в $f$ это произведение количества путей между парами вершин.

Пусть множество $|A| = n$.

Тогда количество строк длины $k$ с элементами из алфавита $A$ которые могут повторятся, это: $n^k$. То же самое, что количество способов упорядоченно выбрать $k$ элементов из множества размера $n$ с повторениями.

Если же повторения запрещены, то количество таких строк это: $\frac{n!}{(n-k)!}$. То же самое, что количество способов упорядоченно выбрать $k$ элементов из множества размера $n$ без повторений.