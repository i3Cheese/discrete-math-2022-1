\subsection{Класс линейных функций, лемма о нелинейной функции.}

\textbf{Функция $f$ называется линейной, если} $f(x_1,\ldots, x_n) = a_0 \oplus a_1x_1 \oplus a_2x_2 \oplus \ldots \oplus a_nx_n$, где $a_i \in \{0, 1\}$

$L = \{f \in P_2 |$ $f$ -- линейная$\}$ -- множество всех линейных функций.

Пример: $x_i \in L$, $x \oplus y \in L$, $0, 1 \in L$

$x \wedge y \notin L, x \vee y \notin L$

\textbf{Лемма о нелинейной функции:} Пусть $f(x_1, \ldots, x_n) \notin L$. Тогда подставив вместо переменных функции $x_1, \ldots, x_n$ 0, $x$ и $y$ можно получить $g(x, y) \notin L$.

\textit{Иначе говоря, через любую не линейную функцию на $n$ переменных можно выразить не линейную функцию на двух переменных.}
